\documentclass[12pt]{article}
\usepackage[margin=5cm]{geometry} 
\usepackage{amsmath,amsthm,amssymb,amsfonts, fancyhdr, color, comment, graphicx, environ}
\usepackage{xcolor}
\usepackage{mdframed}
\usepackage[shortlabels]{enumitem}
\usepackage{indentfirst}
\usepackage{hyperref}
\usepackage{tensor}

\usepackage{logicproof}


\hypersetup{
    colorlinks=true,
    linkcolor=blue,
    filecolor=magenta,      
    urlcolor=blue,
}
\usepackage{tikz-cd}
\tikzset{ampersand replacement = \&}
\pagestyle{fancy}

\usepackage{tikz}
\newcommand*\circled[1]{\tikz[baseline=(char.base)]{\node[shape=circle,draw,inner sep=1pt] (char) {#1};}}
            
    

\newenvironment{problem}[2][Problem]
    { \begin{mdframed}[backgroundcolor=gray!20] \textbf{#1 #2} \\}
    {  \end{mdframed}}

\newenvironment{solution}
    {{\textbf{Solution}}}

\def\boxit#1#2{%
    \smash{\color{red}\fboxrule=1pt\relax\fboxsep=2pt\relax%
    \llap{\rlap{\fbox{\phantom{\rule{#1}{#2}}}}~}}\ignorespaces
}

\newcommand\LL[1]{\multicolumn{1}{|c}{#1}}
\newcommand\RR[1]{\multicolumn{1}{c|}{#1}}
\newcommand\LR[1]{\multicolumn{1}{|c|}{#1}}
% \cline{2-7}


%%%%%%%%%%%%%%%%%%%%%%%%%%%%%%%%%%%%%%%%%%%%%%%
\usepackage{tikzpagenodes}
\usetikzlibrary{calc}

\makeatletter
\tikzset{%
  remember picture with id/.style={%
    remember picture,
    overlay,
    save picture id=#1,
  },
  save picture id/.code={%
    \edef\pgf@temp{#1}%
    \immediate\write\pgfutil@auxout{%
      \noexpand\savepointas{\pgf@temp}{\pgfpictureid}}%
  },
  if picture id/.code args={#1#2#3}{%
    \@ifundefined{save@pt@#1}{%
      \pgfkeysalso{#3}%
    }{
      \pgfkeysalso{#2}%
    }
  }
}

\def\savepointas#1#2{%
  \expandafter\gdef\csname save@pt@#1\endcsname{#2}%
}

\def\tmk@labeldef#1,#2\@nil{%
  \def\tmk@label{#1}%
  \def\tmk@def{#2}%
}

\tikzdeclarecoordinatesystem{pic}{%
  \pgfutil@in@,{#1}%
  \ifpgfutil@in@%
    \tmk@labeldef#1\@nil
  \else
    \tmk@labeldef#1,(0pt,0pt)\@nil
  \fi
  \@ifundefined{save@pt@\tmk@label}{%
    \tikz@scan@one@point\pgfutil@firstofone\tmk@def
  }{%
  \pgfsys@getposition{\csname save@pt@\tmk@label\endcsname}\save@orig@pic%
  \pgfsys@getposition{\pgfpictureid}\save@this@pic%
  \pgf@process{\pgfpointorigin\save@this@pic}%
  \pgf@xa=\pgf@x
  \pgf@ya=\pgf@y
  \pgf@process{\pgfpointorigin\save@orig@pic}%
  \advance\pgf@x by -\pgf@xa
  \advance\pgf@y by -\pgf@ya
  }%
}
\newcommand\tikzmark[2][]{%
\tikz[remember picture with \usepackage{logicproof}
id=#2] #1;}
\makeatother

\newcommand\BoxedText[3][]{%
\begin{tikzpicture}[remember picture,overlay]
\draw[#1] 
  let \p1=(pic cs:#2), \p2=(pic cs:#3) in
  ([yshift=-0.8ex]\p1) --
  ([yshift=2ex]\p1) -- 
  ([xshift=3pt,yshift=2ex]\p1-|current page text area.east) -- 
  ([xshift=3pt,yshift=2ex]\p2-|current page text area.east) --
  ([yshift=2ex]\p2) --
  ([yshift=-0.8ex]\p2) --
  ([xshift=-3pt,yshift=-0.8ex]\p2-|current page text area.west) --
  ([xshift=-3pt,yshift=-0.8ex]\p1-|current page text area.west) --
  cycle
;
\end{tikzpicture}%
}

\newcommand{\gradientbox}[2]{%
    \BoxedText[draw=orange!70!black,right color=orange!10,left color=orange!50]     {start#1}{end#1}
    \tikzmark{start#1}#2
    \tikzmark{end#1}
}

\newcommand{\shadedbox}[2]{%
    \BoxedText[draw=cyan!70!black,fill=cyan!30,ultra thick]{start#1}{end#1}
    \tikzmark{start#1}#2
    \tikzmark{end#1}
}

\newcommand{\outlinebox}[2]{%
    \BoxedText{start#1}{end#1}
    \tikzmark{start#1}#2
    \tikzmark{end#1}
}

\newcommand{\sentence}[4]{$^{\circled{#2}}${\BoxedText[draw=#3!70!black,fill=#3!15,thick]{start#1}{end#1}
    \tikzmark{start#1}#4
    \tikzmark{end#1}}}

\usetikzlibrary{arrows}
\usetikzlibrary{shapes}
\newcommand{\indicator}[1]{%
  \tikz[baseline=(char.base)]\node[anchor=south west, draw,rectangle, rounded corners, inner sep=2pt, minimum size=6mm,
    text height=2mm](char){\textbf{#1}} ;}






\newcommand{\ptext}[1]{\parbox{.9\textwidth}{#1}}

\newcommand{\wff}
{Well Formed Formula }

\def\checkmark{\tikz\fill[scale=0.4](0,.35) -- (.25,0) -- (1,.7) -- (.25,.15) -- cycle;} 

%%%%%%%%%%%%%%%%%%%%%%%%%%%%%%%%%%%%%%%%%%%%%
\lhead{PHIL131.01 \\ Assignment-5}
\rhead{Atahan \\ Haznedar} 
\chead{\textbf{2021107075}}
%%%%%%%%%%%%%%%%%%%%%%%%%%%%%%%%%%%%%%%%%%%%%


\begin{document}
    \section*{Chapter 5}
    \paragraph{I}
    The following arguments are all valid. Formalize each, using the interpretation indicated below, and prove the validity of the resulting form, using only the ten introduction and elimination rules.
    \begin{center}
        \begin{tabular}{l | l}
            Sentence Letter  & Interpretation \\ \hline
            C & The conclusion of this argument is true. \\
            P & The premises of this argument are all false. \\ 
            S & This argument is sound. \\
            V & This argument is valid. \\
        \end{tabular}
    \end{center}
    
    \begin{problem}{1}
        This argument is not unsound. Therefore, this argument is sound.
    \end{problem}
    
        \begin{solution}
            \begin{logicproof}{1}
                \sim(\sim S) & A \\
                S & 1 $\sim$ E
            \end{logicproof}
        \end{solution}
    
    
    \begin{problem}{2}
        This argument is sound. Therefore, this argument is not unsound.
    \end{problem}
        
        \begin{solution}
            \begin{logicproof}{2}
                S & A\\
                \sim\sim S & 1DN
            \end{logicproof}
        \end{solution}
\clearpage
    \begin{problem}{6}
        This argument is both sound and valid Therefore, either it is sound or it is invalid
    \end{problem}
        
        \begin{solution}
            \begin{logicproof}{2}
                S \& V & A \\
                S & 1\&E \\
                $S \lor \neg V$ & $2\lor I$
            \end{logicproof}        
        \end{solution}
    
    \begin{problem}{7}
        This argument is not both sound and invalid. It is sound. Therefore, it is valid.
    \end{problem}
        
        \begin{solution}
            \begin{logicproof}{4}
                $\neg(S \& \neg V)$ & $A$ \\
                $S$ & $A$ \\
                $\neg S \lor \neg \neg V$ & $1DM$ \\
                $S \lor \neg \neg V$ & $2\lor I$ \\
                $(\neg S \lor \neg \neg V) \lor (S \lor \neg \neg V)$ & $3,4 \&$ I \\
                $\neg \neg V \lor (\neg S \& S)$ & $5DIST$\\
                $\neg \neg V$ & $6CON$ \\
                $V$ & $7 \neg E$
            \end{logicproof}
        \end{solution}

\clearpage

    \begin{problem}{8}
        This argument is sound only if all its premises are true. But not all its premises are true. Hence it is unsound.
    \end{problem}
    
        \begin{solution}
            \begin{logicproof}{1}
                $S \Rightarrow P$ & $A$ \\
                $\neg P$ & $A$ \\
                $\neg P \Rightarrow \neg S$ & $1$ TRANS \\
                $\neg S $ & $2,3 \rightarrow E$
            \end{logicproof}
        \end{solution}
    
    \begin{problem}{12}
        Either this argument is unsound, or else i is valid and all its premises are true. Hence it is either unsound or valid.
    \end{problem}
        
        \begin{solution}
            \begin{logicproof}{7}
                $\sim S \lor (V \& P)$ & A \\
                \begin{subproof}
                    $\sim S$ & H \\
                    $\sim S \lor V$ & $2 \lor I$
                \end{subproof}
                $\sim S \rightarrow (\sim S \lor V)$ & $2,3\rightarrow I$\\
                \begin{subproof}
                    $V \& P$ & H \\
                    $V$ & $5\&E$ \\
                    $\sim S \lor V$ & $6 \lor I$ 
                \end{subproof}
                $(V \& P) \rightarrow (\sim S \lor V)$ & $5,7 \rightarrow I$ \\
                $\sim S \lor V $ & $1,4,8\lor E$
            \end{logicproof}
        \end{solution}
    
\end{document}