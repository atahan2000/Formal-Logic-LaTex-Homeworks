\documentclass[12pt]{article}
\usepackage[margin=5cm]{geometry} 
\usepackage{amsmath,amsthm,amssymb,amsfonts, fancyhdr, color, comment, graphicx, environ}
\usepackage{xcolor}
\usepackage{mdframed}
\usepackage[shortlabels]{enumitem}
\usepackage{indentfirst}
\usepackage{hyperref}
\usepackage{tensor}
\hypersetup{
    colorlinks=true,
    linkcolor=blue,
    filecolor=magenta,      
    urlcolor=blue,
}
\usepackage{tikz-cd}
\tikzset{ampersand replacement = \&}
\pagestyle{fancy}

\usepackage{tikz}
\newcommand*\circled[1]{\tikz[baseline=(char.base)]{\node[shape=circle,draw,inner sep=1pt] (char) {#1};}}
            
    

\newenvironment{problem}[2][Problem]
    { \begin{mdframed}[backgroundcolor=gray!20] \textbf{#1 #2} \\}
    {  \end{mdframed}}

\newenvironment{solution}
    {{\textbf{Solution}}}

\def\boxit#1#2{%
    \smash{\color{red}\fboxrule=1pt\relax\fboxsep=2pt\relax%
    \llap{\rlap{\fbox{\phantom{\rule{#1}{#2}}}}~}}\ignorespaces
}

\newcommand\LL[1]{\multicolumn{1}{|c}{#1}}
\newcommand\RR[1]{\multicolumn{1}{c|}{#1}}
\newcommand\LR[1]{\multicolumn{1}{|c|}{#1}}
% \cline{2-7}


%%%%%%%%%%%%%%%%%%%%%%%%%%%%%%%%%%%%%%%%%%%%%%%
\usepackage{tikzpagenodes}
\usetikzlibrary{calc}

\makeatletter
\tikzset{%
  remember picture with id/.style={%
    remember picture,
    overlay,
    save picture id=#1,
  },
  save picture id/.code={%
    \edef\pgf@temp{#1}%
    \immediate\write\pgfutil@auxout{%
      \noexpand\savepointas{\pgf@temp}{\pgfpictureid}}%
  },
  if picture id/.code args={#1#2#3}{%
    \@ifundefined{save@pt@#1}{%
      \pgfkeysalso{#3}%
    }{
      \pgfkeysalso{#2}%
    }
  }
}

\def\savepointas#1#2{%
  \expandafter\gdef\csname save@pt@#1\endcsname{#2}%
}

\def\tmk@labeldef#1,#2\@nil{%
  \def\tmk@label{#1}%
  \def\tmk@def{#2}%
}

\tikzdeclarecoordinatesystem{pic}{%
  \pgfutil@in@,{#1}%
  \ifpgfutil@in@%
    \tmk@labeldef#1\@nil
  \else
    \tmk@labeldef#1,(0pt,0pt)\@nil
  \fi
  \@ifundefined{save@pt@\tmk@label}{%
    \tikz@scan@one@point\pgfutil@firstofone\tmk@def
  }{%
  \pgfsys@getposition{\csname save@pt@\tmk@label\endcsname}\save@orig@pic%
  \pgfsys@getposition{\pgfpictureid}\save@this@pic%
  \pgf@process{\pgfpointorigin\save@this@pic}%
  \pgf@xa=\pgf@x
  \pgf@ya=\pgf@y
  \pgf@process{\pgfpointorigin\save@orig@pic}%
  \advance\pgf@x by -\pgf@xa
  \advance\pgf@y by -\pgf@ya
  }%
}
\newcommand\tikzmark[2][]{%
\tikz[remember picture with id=#2] #1;}
\makeatother

\newcommand\BoxedText[3][]{%
\begin{tikzpicture}[remember picture,overlay]
\draw[#1] 
  let \p1=(pic cs:#2), \p2=(pic cs:#3) in
  ([yshift=-0.8ex]\p1) --
  ([yshift=2ex]\p1) -- 
  ([xshift=3pt,yshift=2ex]\p1-|current page text area.east) -- 
  ([xshift=3pt,yshift=2ex]\p2-|current page text area.east) --
  ([yshift=2ex]\p2) --
  ([yshift=-0.8ex]\p2) --
  ([xshift=-3pt,yshift=-0.8ex]\p2-|current page text area.west) --
  ([xshift=-3pt,yshift=-0.8ex]\p1-|current page text area.west) --
  cycle
;
\end{tikzpicture}%
}

\newcommand{\gradientbox}[2]{%
    \BoxedText[draw=orange!70!black,right color=orange!10,left color=orange!50]     {start#1}{end#1}
    \tikzmark{start#1}#2
    \tikzmark{end#1}
}

\newcommand{\shadedbox}[2]{%
    \BoxedText[draw=cyan!70!black,fill=cyan!30,ultra thick]{start#1}{end#1}
    \tikzmark{start#1}#2
    \tikzmark{end#1}
}

\newcommand{\outlinebox}[2]{%
    \BoxedText{start#1}{end#1}
    \tikzmark{start#1}#2
    \tikzmark{end#1}
}

\newcommand{\sentence}[4]{$^{\circled{#2}}${\BoxedText[draw=#3!70!black,fill=#3!15,thick]{start#1}{end#1}
    \tikzmark{start#1}#4
    \tikzmark{end#1}}}

\usetikzlibrary{arrows}
\usetikzlibrary{shapes}
\newcommand{\indicator}[1]{%
  \tikz[baseline=(char.base)]\node[anchor=south west, draw,rectangle, rounded corners, inner sep=2pt, minimum size=6mm,
    text height=2mm](char){\textbf{#1}} ;}


\newcommand{\ptext}[1]{\parbox{.9\textwidth}{#1}}

\newcommand{\wff}
{Well Formed Formula }
%%%%%%%%%%%%%%%%%%%%%%%%%%%%%%%%%%%%%%%%%%%%%
\lhead{PHIL131.01 \\ Assignment-1}
\rhead{Atahan \\ Haznedar} 
\chead{\textbf{2021107075}}
%%%%%%%%%%%%%%%%%%%%%%%%%%%%%%%%%%%%%%%%%%%%%


\begin{document}
    \section*{Chapter 3}
    \paragraph{I.}
    Formalize the following statements, using the interpretation indicated below: \\
    
    \begin{tabular}{|l | l|}
    \hline
        Sentence Letter & Interpretation \\ \hline
        P & Pam is going \\
        Q & Quincy is going. \\
        R & Richard is going. \\
        S & Sara is going. \\ \hline
    \end{tabular}
    
    \begin{problem}{(1)}
        Pam is not going.
    \end{problem}
    \begin{solution}
        $\sim P$
    \end{solution}
    \begin{problem}{(2)}
        Pam is going, but Quincy is not.
    \end{problem}
    \begin{solution}
        $(P \land \sim Q)$
    \end{solution}
    \begin{problem}{(3)}
        If Pam is going, then so is Quincy.
    \end{problem}
    \begin{solution}
        $(P \Rightarrow Q)$
    \end{solution}
    \begin{problem}{(4)}
        Pam is going if Quincy is.
    \end{problem}
    \begin{solution}
        $(Q \Rightarrow P)$
    \end{solution}
    \begin{problem}{(5)}
        Pam is going only if Quincy is.
    \end{problem}
    \begin{solution}
        $(Q \iff P)$
    \end{solution}
    \begin{problem}{(6)}
        Pam is going if and only if Quincy is.
    \end{problem}
    \begin{solution}
        $(P \iff Q)$
    \end{solution}
    \begin{problem}{(7)}
        Neither Pam nor Quincy is going.
    \end{problem}
    \begin{solution}
        $(\sim P \land  \sim Q)$
    \end{solution}
    \begin{problem}{(8)}
        Pam and Quincy are not both going.
    \end{problem}
    \begin{solution}
        $\sim(P \land Q)$
    \end{solution}
    \begin{problem}{(9)}
        Either Pam is not going or Quincy is not going.
    \end{problem}
    \begin{solution}
        $(\sim P \lor \sim Q)$
    \end{solution}
    \begin{problem}{(10)}
        Pam is not going if Quincy is.
    \end{problem}
    \begin{solution}
        $(Q \Rightarrow \sim P)$
    \end{solution}
    \begin{problem}{(11)}
        Either Pam is going, or Richard and Quincy are going.
    \end{problem}
    \begin{solution}
        $(P \lor (R \land Q))$
    \end{solution}
    \begin{problem}{(12)}
        If Pam is going, then both Richard and Quincy are going.
    \end{problem}
    \begin{solution}
        $(P \Rightarrow (R \land Q))$
    \end{solution}
    \begin{problem}{(13)}
        Pam is staying, but Richard and Quincy are going.
    \end{problem}
    \begin{solution}
        $(\sim P \land (R \land Q))$
    \end{solution}
    \begin{problem}{(14)}
        If Richard is going, then if Pam is staying, Quincy is going.
    \end{problem}
    \begin{solution}
        $(R \Rightarrow (\sim P \Rightarrow Q))$
    \end{solution} 
    \begin{problem}{(15)}
        If neither Richard nor Quincy is going, then Pam is going.
    \end{problem}
    \begin{solution}
        $((\sim R \& \sim Q) \Rightarrow P)$
    \end{solution}
    \begin{problem}{(16)}
        Richard is going only if Pam and Quincy are staying.
    \end{problem}
    \begin{solution}
        $(R \iff (\sim P \land \sim Q ))$
    \end{solution}
    \begin{problem}{(17)}
        Richard and Quincy are going, although Pam and Sara are staying.
    \end{problem}
    \begin{solution}
        $((R \land Q) \land (\sim P \land \sim S))$
    \end{solution}
    \begin{problem}{(18)}
        If either Richard or Quincy is going, then Pam is going and Sara is staying.
    \end{problem}
    \begin{solution}
        $((R \lor Q) \Rightarrow (P \land \sim S))$
    \end{solution}
    \begin{problem}{(19)}
        Richard and Quincy are going if and only if either Pam or Sara is going.
    \end{problem}
    \begin{solution}
        $((R \land Q) \iff (P \lor S))$
    \end{solution}
    \begin{problem}{(20)}
        If Sara is going, then either Richard or Pam is going, and if Sara is not going, then both Pam and Quincy are going.
    \end{problem}
    \begin{solution}
        $((S \Rightarrow (R \lor P)) \land  (\sim S \Rightarrow (P \land Q)))$
    \end{solution}
    
\paragraph{II.}
Determine which of the following formulas are \wff and which are not. Explain your solution. \\

{\bf Formation Rules of \wff}
    \begin{enumerate}
        \item Any sentence letter is a \wff
        \item If $\phi$ is a \wff, then so is $\sim\phi$
        \item If $\phi$ and $\psi$ are \wff's, then so are $(\phi \& \psi)$, $(\phi \lor \psi)$,$(\phi \Rightarrow \psi)$, $(\phi \iff \psi)$
    \end{enumerate}    
    
    \begin{problem}{(1)}
        $\sim(\sim P)$
    \end{problem}
    
    \begin{solution}
        Its not a \wff since, there is no need to put 'paranthesis' after '$\sim$'(Negation)
    \end{solution}
    
    \begin{problem}{(2)}
        $P \sim Q$
    \end{problem}
    
    \begin{solution}
        Its not a \wff since, '$\sim$' is an unary operation.
    \end{solution}
    
    \begin{problem}{(3)}
        $(P \Rightarrow P)$
    \end{problem}
    
    \begin{solution}
        Its a \wff.
    \end{solution}
    
    \begin{problem}{(4)}
        $P \Rightarrow P$ 
    \end{problem}
    
    \begin{solution}
        Its not a \wff since it omits the 3rd law of 'Rules of Formation'. There should be paranthesis in both ends.
    \end{solution}
    
    \begin{problem}{(5)}
        $\sim \sim \sim (\sim P \& Q)$
    \end{problem}
    
    \begin{solution}
        It is a \wff.
    \end{solution}
    
    \begin{problem}{(6)}
        $(( P \Rightarrow Q))$
    \end{problem}
    
    \begin{solution}
        It is not a \wff since it omits the 3rd law of 'Rules of Formation'. The outer paranthesis shouldn't have placed there.
    \end{solution}
    
    \begin{problem}{(7)}
        $\sim (P \& Q) \& \sim R$
    \end{problem}
    
    \begin{solution}
        It is not a \wff since it omits the 3rd law of 'Rules of Formation'. There should be another paranthesis both at the end and at the beginning.
    \end{solution}
    
    \begin{problem}{(8)}
        $(P \iff (P \iff (P \iff P)))$
    \end{problem}
    
    \begin{solution}
        It is a \wff.
    \end{solution}
    
    \begin{problem}{(9)}
        $(P \Rightarrow (Q \Rightarrow (R \& S))$
    \end{problem}
    
    \begin{solution}
        It is not a \wff since it omits the 3rd law of 'Rules of Formation'. There should be another paranthes at the end.
    \end{solution}
    
    \begin{problem}{(10)}
        $(P \Rightarrow (Q \lor R \lor S))$
    \end{problem}
    
    \begin{solution}
        It is not a \wff since it omits the 3rd law of 'Rules of Formation'. It should have been written as $$ (P \Rightarrow ((Q \lor R) \lor (R \lor S)))$$
    \end{solution}

\paragraph{III.}
Determine whether the following formulas are tautologous, truth-functionally contingent, or inconsistent, using either truth tables or refutation trees.

\begin{problem}{(1)}
    $P \Rightarrow P$
\end{problem}

\begin{solution}
    It is a Tautology.
    \begin{center}
        \begin{tabular}{c|c c c}
            P & (P & $\Rightarrow$ & P) \\ \hline
            T & T & \LR{T} & T \\
            F & F & \LR{T} & F \\ \cline{3-3}
        \end{tabular}
    \end{center}
\end{solution}

\begin{problem}{(2)}
    $P \Rightarrow \sim P$
\end{problem}

\begin{solution}
It is a contingent formula.
    \begin{center}
        \begin{tabular}{c|c c c c}
            P & (P & $\Rightarrow$ & $ \sim$ & P) \\ \hline
            T & T & \LR{F} & F & T \\
            F & F & \LR{T} & T & F \\ \cline{3-3}
        \end{tabular}
    \end{center}
\end{solution}

\begin{problem}{(3)}
    $\sim(P \Rightarrow P)$
\end{problem}

\begin{solution}
    It is inconsistent.
    \begin{center}
        \begin{tabular}{c|c c c c}
            P & $\sim$ &(P & $\Rightarrow$ &  P) \\ \hline
            T & \LR{F} & T & T & T \\
            F & \LR{F} & F & T & F \\  \cline{2-2}
        \end{tabular}
    \end{center}
\end{solution}

\begin{problem}{(4)}
    $P \Rightarrow Q$
\end{problem}

\begin{solution}
    It is a Tautology.
    \begin{center}
        \begin{tabular}{c c|c c c}
            P & Q &(P & $\Rightarrow$ & Q) \\ \hline
            T & T & T & \LR{T} & T \\
            F & F & T & \LR{T} & F \\
            F & T & F & \LR{T} & T \\
            F & F & F & \LR{T} & F \\ \cline{4-4}
        \end{tabular} 
    \end{center}
\end{solution}


\begin{problem}{(5)}
    $(P \lor Q) \Rightarrow P$
\end{problem}

\begin{solution}
    It is contingent formula.
    \begin{center}
        \begin{tabular}{c c|c c c c c}
            P & Q &(P & $\lor$ & Q) & $\Rightarrow$ & P \\ \hline
            T & T & T & T & T & \LR{T} & T \\
            T & F & T & T & F & \LR{T} & T \\
            F & T & F & T & T & \LR{F} & F \\
            F & F & F & F & F & \LR{T} & F \\ \cline{6-6}
        \end{tabular}
    \end{center}
\end{solution}

\begin{problem}{(6)}
    $(P \& Q) \Rightarrow P$
\end{problem}

\begin{solution}
    It is a tautology.
    \begin{center}
        \begin{tabular}{c c|c c c c c}
            P & Q &(P & $\&$ & Q) & $\Rightarrow$ & P \\ \hline
            T & T & T & T & T & \LR{T} & T \\
            T & F & T & F & F & \LR{T} & T \\
            F & T & F & F & T & \LR{T} & F \\
            F & F & F & F & F & \LR{T} & F \\ \cline{6-6}
        \end{tabular}
    \end{center}
\end{solution}

\begin{problem}{(7)}
    $P \iff \sim(P \lor Q)$
\end{problem}

\begin{solution}
    It is a contingent formula.
    \begin{center}
        \begin{tabular}{c c|c c c c c c}
            P & Q & P & $\iff$ & $\sim$( & P & $\lor$ & Q) \\ \hline
            T & T & T & \LR{F} & F & T & T & T \\
            T & F & T & \LR{F} & F & T & T & F\\
            F & T & F & \LR{T} & F & F & T & T\\
            F & F & F & \LR{F} & T & F & F & F\\ \cline{4-4}
        \end{tabular}
    \end{center}
\end{solution}

\begin{problem}{(8)}
    $\sim ((P \& Q) \iff (P \lor Q))$
\end{problem}

\begin{solution}
    It is a contingent formula.
    \begin{center}
        \begin{tabular}{c c|c c c c c c c c}
            P & Q & $\sim$ & ((P & \& & Q) & $\iff$ & (P & $\lor$ & Q))) \\ \hline
            T & T & \LR{F} & T & T & T & T & T & T & T \\
            T & F & \LR{T} & T & F & F & F & T & T & F\\
            F & T & \LR{T} & F & F & T & F & F & T & T\\
            F & F & \LR{F} & F & F & F & T & F & F & F\\ \cline{3-3}
        \end{tabular}
    \end{center}
\end{solution}

\begin{problem}{(9)}
    $(P \& Q) \& \sim(P \lor R)$
\end{problem}

\begin{solution}
    It is inconsistent.
    \begin{center}
        \begin{tabular}{c c c|c c c c c c c c}
            P & Q & R & (P & \& & Q) & \& & $\sim$( & P & $\lor$ & R) \\ \hline
            T & T & T & T & T & T & \LR{F} & F & T & T & T \\
            T & T & F & T & T & T & \LR{F} & F & T & T & F\\
            T & F & T & T & F & F & \LR{F} & F & T & T & T\\
            T & F & F & T & F & F & \LR{F} & F & T & T & F\\
            F & T & T & F & F & T & \LR{F} & F & F & T & T\\
            F & T & F & F & F & T & \LR{F} & T & F & F & F\\
            F & F & T & F & F & F & \LR{F} & F & F & T & T\\
            F & F & F & F & F & F & \LR{F} & T & F & F & F\\ \cline{7-7}
        \end{tabular}
    \end{center}
\end{solution}

\begin{problem}{(10)}
    $(P \Rightarrow (Q \& R)) \Rightarrow (P \Rightarrow R)$
\end{problem}

\begin{solution}
    It is a tautology.
    \begin{center}
        \begin{tabular}{c c c|c c c c c c c c c}
            P & Q & R & (P & $\Rightarrow$ & (Q & \& & R)) & $\Rightarrow$ & (P & $\Rightarrow$ & R) \\ \hline
            T & T & T & T & T & T & T & T & \LR{T} & T & T & T\\
            T & T & F & T & F & T & F & F & \LR{T} & T & F & F\\
            T & F & T & T & F & F & F & T & \LR{T} & T & T & T\\
            T & F & F & T & F & F & F & F & \LR{T} & T & F & F\\
            F & T & T & F & T & T & T & T & \LR{T} & F & T & T\\
            F & T & F & F & T & T & F & F & \LR{T} & F & T & T\\
            F & F & T & F & T & F & F & T & \LR{T} & F & T & T\\
            F & F & F & F & T & F & F & F & \LR{T} & F & T & F\\ \cline{9-9}
        \end{tabular}
    \end{center}
\end{solution}

\end{document}