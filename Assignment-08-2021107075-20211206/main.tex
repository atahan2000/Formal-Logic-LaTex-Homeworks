w\documentclass[12pt]{article}
\usepackage[margin=5cm]{geometry} 
\usepackage{amsmath,amsthm,amssymb,amsfonts, fancyhdr, color, comment, graphicx, environ}
\usepackage{xcolor}
\usepackage{mdframed}
\usepackage[shortlabels]{enumitem}
\usepackage{indentfirst}
\usepackage{hyperref}
\usepackage{tensor}

\usepackage{logicproof}

\usepackage{graphicx}
\graphicspath{ {./images/} }

\hypersetup{
    colorlinks=true,
    linkcolor=blue,
    filecolor=magenta,      
    urlcolor=blue,
}
\usepackage{tikz-cd}
\tikzset{ampersand replacement = \&}
\pagestyle{fancy}

\usepackage{tikz}
\newcommand*\circled[1]{\tikz[baseline=(char.base)]{\node[shape=circle,draw,inner sep=1pt] (char) {#1};}}
            
    

\newenvironment{problem}[2][Problem]
    { \begin{mdframed}[backgroundcolor=gray!20] \textbf{#1 #2} \\}
    {  \end{mdframed}}

\newenvironment{solution}
    {{\textbf{Solution}}}

\def\boxit#1#2{%
    \smash{\color{red}\fboxrule=1pt\relax\fboxsep=2pt\relax%
    \llap{\rlap{\fbox{\phantom{\rule{#1}{#2}}}}~}}\ignorespaces
}

\newcommand\LL[1]{\multicolumn{1}{|c}{#1}}
\newcommand\RR[1]{\multicolumn{1}{c|}{#1}}
\newcommand\LR[1]{\multicolumn{1}{|c|}{#1}}
% \cline{2-7}


%%%%%%%%%%%%%%%%%%%%%%%%%%%%%%%%%%%%%%%%%%%%%%%
\usepackage{tikzpagenodes}
\usetikzlibrary{calc}

\makeatletter
\tikzset{%
  remember picture with id/.style={%
    remember picture,
    overlay,
    save picture id=#1,
  },
  save picture id/.code={%
    \edef\pgf@temp{#1}%
    \immediate\write\pgfutil@auxout{%
      \noexpand\savepointas{\pgf@temp}{\pgfpictureid}}%
  },
  if picture id/.code args={#1#2#3}{%
    \@ifundefined{save@pt@#1}{%
      \pgfkeysalso{#3}%
    }{
      \pgfkeysalso{#2}%
    }
  }
}

\def\savepointas#1#2{%
  \expandafter\gdef\csname save@pt@#1\endcsname{#2}%
}

\def\tmk@labeldef#1,#2\@nil{%
  \def\tmk@label{#1}%
  \def\tmk@def{#2}%
}

\tikzdeclarecoordinatesystem{pic}{%
  \pgfutil@in@,{#1}%
  \ifpgfutil@in@%
    \tmk@labeldef#1\@nil
  \else
    \tmk@labeldef#1,(0pt,0pt)\@nil
  \fi
  \@ifundefined{save@pt@\tmk@label}{%
    \tikz@scan@one@point\pgfutil@firstofone\tmk@def
  }{%
  \pgfsys@getposition{\csname save@pt@\tmk@label\endcsname}\save@orig@pic%
  \pgfsys@getposition{\pgfpictureid}\save@this@pic%
  \pgf@process{\pgfpointorigin\save@this@pic}%
  \pgf@xa=\pgf@x
  \pgf@ya=\pgf@y
  \pgf@process{\pgfpointorigin\save@orig@pic}%
  \advance\pgf@x by -\pgf@xa
  \advance\pgf@y by -\pgf@ya
  }%
}
\newcommand\tikzmark[2][]{%
\tikz[remember picture with \usepackage{logicproof}
id=#2] #1;}
\makeatother

\newcommand\BoxedText[3][]{%
\begin{tikzpicture}[remember picture,overlay]
\draw[#1] 
  let \p1=(pic cs:#2), \p2=(pic cs:#3) in
  ([yshift=-0.8ex]\p1) --
  ([yshift=2ex]\p1) -- 
  ([xshift=3pt,yshift=2ex]\p1-|current page text area.east) -- 
  ([xshift=3pt,yshift=2ex]\p2-|current page text area.east) --
  ([yshift=2ex]\p2) --
  ([yshift=-0.8ex]\p2) --
  ([xshift=-3pt,yshift=-0.8ex]\p2-|current page text area.west) --
  ([xshift=-3pt,yshift=-0.8ex]\p1-|current page text area.west) --
  cycle
;
\end{tikzpicture}%
}

\newcommand{\gradientbox}[2]{%
    \BoxedText[draw=orange!70!black,right color=orange!10,left color=orange!50]     {start#1}{end#1}
    \tikzmark{start#1}#2
    \tikzmark{end#1}
}

\newcommand{\shadedbox}[2]{%
    \BoxedText[draw=cyan!70!black,fill=cyan!30,ultra thick]{start#1}{end#1}
    \tikzmark{start#1}#2
    \tikzmark{end#1}
}

\newcommand{\outlinebox}[2]{%
    \BoxedText{start#1}{end#1}
    \tikzmark{start#1}#2
    \tikzmark{end#1}
}

\newcommand{\sentence}[4]{$^{\circled{#2}}${\BoxedText[draw=#3!70!black,fill=#3!15,thick]{start#1}{end#1}
    \tikzmark{start#1}#4
    \tikzmark{end#1}}}

\usetikzlibrary{arrows}
\usetikzlibrary{shapes}
\newcommand{\indicator}[1]{%
  \tikz[baseline=(char.base)]\node[anchor=south west, draw,rectangle, rounded corners, inner sep=2pt, minimum size=6mm,
    text height=2mm](char){\textbf{#1}} ;}






\newcommand{\ptext}[1]{\parbox{.9\textwidth}{#1}}

\newcommand{\wff}
{Well Formed Formula }

\def\checkmark{\tikz\fill[scale=0.4](0,.35) -- (.25,0) -- (1,.7) -- (.25,.15) -- cycle;} 

%%%%%%%%%%%%%%%%%%%%%%%%%%%%%%%%%%%%%%%%%%%%%
\lhead{PHIL131.01 \\ Assignment-8}
\rhead{Atahan \\ Haznedar} 
\chead{05318867572 \\\textbf{2021107075}}
%%%%%%%%%%%%%%%%%%%%%%%%%%%%%%%%%%%%%%%%%%%%%


\begin{document}
    \section*{Chapter 6}
    \paragraph{I}
    Formalize the following English sentences, using the interpretation given below. (The last five require the  identity predicate.) \\[2pt]
        \begin{tabular}{c |l}
             \emph{Symbol} & \emph{Interpretation} \\ \hline
             Names & \\
             \emph{a} & Al \\
             \emph{b} & Beth \\
             \emph{f} & fame \\
             \emph{m} & money \\
             & \\
             One-Place Predicates & \\
             \emph{F} & is famous \\
             \emph{G} & is greedy \\
             \emph{H} & is a human being \\
             Two-Place Predicate & \\
             \emph{L} & likes \\
             Three-Place Predicate & \\
             \emph{P} & prefers...to
        \end{tabular}
    
    
    
    
    
    \begin{problem}{(1)}
        Both Al and Beth like money. 
    \end{problem}
    
    \begin{solution}
         $L_{am}\&L_{bm}$
    \end{solution}
    
    \begin{problem}{(2)}
        Neither Al nor Beth is famous.     
    \end{problem}
    
    \begin{solution}
        $\sim F_a \&\sim F_b$
    \end{solution}
    
    
    \begin{problem}{(3)}
        Al likes both fame and money.     
    \end{problem}
    
    \begin{solution}
        $L_{af}\&L_{am}$
    \end{solution}
    
    \begin{problem}{(4)}
        Beth prefers fame to money.    
    \end{problem}
    
    
    \begin{solution}
        $P_{bfm}$
    \end{solution}
    
    \begin{problem}{(6)}
        Beth prefers something to Al.   
    \end{problem}
    
    \begin{solution}
        $\exists x P_{bxa}$
    \end{solution}
    
    \begin{problem}{(7)}
        Al prefers nothing to Beth.         
    \end{problem}
    
    \begin{solution}
        $\sim \exists P_{axb}$
    \end{solution}
    
    \begin{problem}{(8)}
        Some humans are both greedy and famous.        
    \end{problem}
    
    \begin{solution}
        $\exists G_x \& F_x \& H_x$
    \end{solution}
    
    \begin{problem}{(9)}
        Anything which is greedy likes money.       
    \end{problem}
    
    \begin{solution}
        $\forall x G_x \rightarrow L_{xm}$
    \end{solution}
    
    \begin{problem}{(11)}
        If Al and Beth are greedy, then a greedy human being exists. 
    \end{problem}
    
    \begin{solution}
        $(G_a \& G_b) \rightarrow (\exists x G_x \& H_x)$
    \end{solution}
    
    \begin{problem}{(12)}
        Beth likes every human being. 
    \end{problem}
    
    \begin{solution}
        $\forall x H_x \rightarrow L_{bx}$
    \end{solution}
    
    \begin{problem}{(13)}
        Something does not like itself.        
    \end{problem}
    
    \begin{solution}
        $\exists x \sim L_{xx}$
    \end{solution}
    
    
    \begin{problem}{(14)}
        Something does not like something.       
    \end{problem}
    
    \begin{solution}
        $\exists x \exists y \sim L_{xy}$
    \end{solution}
    
    \begin{problem}{(16)}
        No human being likes everything.     
    \end{problem}
 
    \begin{solution}
        $\sim \exists x (H_x \& \forall y L_{xy})$
    \end{solution}
     
    \begin{problem}{(17)}
        Al likes every human being who likes him.     
    \end{problem}
    
    \begin{solution}
        $\forall x (H_x \& L_{xa}) \rightarrow L_{ax}$
    \end{solution}
    
    \begin{problem}{(18)}
        Some of the famous like themselves.         
    \end{problem}
    
    \begin{solution}
        $\exists x Fx \& L_{xx} $
    \end{solution}
    
    \begin{problem}{(19)}
        The famous do not all like themselves.        
    \end{problem}
    
    \begin{solution}
        I interpreted it as "Not all famouses like themselves." $$\exists x F_x \& \sim L_{xx}$$
    \end{solution}
    
    \begin{problem}{(21)}
        If Al is greedy and Beth is not, then Al and Beth are not identical.         
    \end{problem}
    
    \begin{solution}
        $(G_a \& \sim G_b) \rightarrow \sim (a =b)$
    \end{solution}
    
    \begin{problem}{(22)}
        Beth likes everything except Al.         
    \end{problem}
    
    \begin{solution}
        $\forall x \sim(x = a) \rightarrow L_{bx}$
    \end{solution}
    
    \begin{problem}{(23)}
         Al is the only human being who is not greedy.        
    \end{problem}
    
    \begin{solution}
        $\forall x (H_x \& \sim (x=a)) \rightarrow G_x$
    \end{solution}
    
    \begin{problem}{(24)}
        Al prefers money to everything else.        
    \end{problem}
    
    \begin{solution}
        $\forall x \sim(x=m)\rightarrow P_{amx}$
    \end{solution}
    

 
    
    
\end{document}