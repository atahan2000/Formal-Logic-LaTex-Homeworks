\documentclass[12pt]{article}
\usepackage[margin=5cm]{geometry} 
\usepackage{amsmath,amsthm,amssymb,amsfonts, fancyhdr, color, comment, graphicx, environ}
\usepackage{xcolor}
\usepackage{mdframed}
\usepackage[shortlabels]{enumitem}
\usepackage{indentfirst}
\usepackage{hyperref}
\usepackage{tensor}

\usepackage{logicproof}

\usepackage{graphicx}
\graphicspath{ {./images/} }

\hypersetup{
    colorlinks=true,
    linkcolor=blue,
    filecolor=magenta,      
    urlcolor=blue,
}
\usepackage{tikz-cd}
\tikzset{ampersand replacement = \&}
\pagestyle{fancy}

\usepackage{tikz}
\newcommand*\circled[1]{\tikz[baseline=(char.base)]{\node[shape=circle,draw,inner sep=1pt] (char) {#1};}}
            
    

\newenvironment{problem}[2][Problem]
    { \begin{mdframed}[backgroundcolor=gray!20] \textbf{#1 #2} \\}
    {  \end{mdframed}}

\newenvironment{solution}
    {{\textbf{Solution}}}

\def\boxit#1#2{%
    \smash{\color{red}\fboxrule=1pt\relax\fboxsep=2pt\relax%
    \llap{\rlap{\fbox{\phantom{\rule{#1}{#2}}}}~}}\ignorespaces
}

\newcommand\LL[1]{\multicolumn{1}{|c}{#1}}
\newcommand\RR[1]{\multicolumn{1}{c|}{#1}}
\newcommand\LR[1]{\multicolumn{1}{|c|}{#1}}
% \cline{2-7}


%%%%%%%%%%%%%%%%%%%%%%%%%%%%%%%%%%%%%%%%%%%%%%%
\usepackage{tikzpagenodes}
\usetikzlibrary{calc}

\makeatletter
\tikzset{%
  remember picture with id/.style={%
    remember picture,
    overlay,
    save picture id=#1,
  },
  save picture id/.code={%
    \edef\pgf@temp{#1}%
    \immediate\write\pgfutil@auxout{%
      \noexpand\savepointas{\pgf@temp}{\pgfpictureid}}%
  },
  if picture id/.code args={#1#2#3}{%
    \@ifundefined{save@pt@#1}{%
      \pgfkeysalso{#3}%
    }{
      \pgfkeysalso{#2}%
    }
  }
}

\def\savepointas#1#2{%
  \expandafter\gdef\csname save@pt@#1\endcsname{#2}%
}

\def\tmk@labeldef#1,#2\@nil{%
  \def\tmk@label{#1}%
  \def\tmk@def{#2}%
}

\tikzdeclarecoordinatesystem{pic}{%
  \pgfutil@in@,{#1}%
  \ifpgfutil@in@%
    \tmk@labeldef#1\@nil
  \else
    \tmk@labeldef#1,(0pt,0pt)\@nil
  \fi
  \@ifundefined{save@pt@\tmk@label}{%
    \tikz@scan@one@point\pgfutil@firstofone\tmk@def
  }{%
  \pgfsys@getposition{\csname save@pt@\tmk@label\endcsname}\save@orig@pic%
  \pgfsys@getposition{\pgfpictureid}\save@this@pic%
  \pgf@process{\pgfpointorigin\save@this@pic}%
  \pgf@xa=\pgf@x
  \pgf@ya=\pgf@y
  \pgf@process{\pgfpointorigin\save@orig@pic}%
  \advance\pgf@x by -\pgf@xa
  \advance\pgf@y by -\pgf@ya
  }%
}
\newcommand\tikzmark[2][]{%
\tikz[remember picture with \usepackage{logicproof}
id=#2] #1;}
\makeatother

\newcommand\BoxedText[3][]{%
\begin{tikzpicture}[remember picture,overlay]
\draw[#1] 
  let \p1=(pic cs:#2), \p2=(pic cs:#3) in
  ([yshift=-0.8ex]\p1) --
  ([yshift=2ex]\p1) -- 
  ([xshift=3pt,yshift=2ex]\p1-|current page text area.east) -- 
  ([xshift=3pt,yshift=2ex]\p2-|current page text area.east) --
  ([yshift=2ex]\p2) --
  ([yshift=-0.8ex]\p2) --
  ([xshift=-3pt,yshift=-0.8ex]\p2-|current page text area.west) --
  ([xshift=-3pt,yshift=-0.8ex]\p1-|current page text area.west) --
  cycle
;
\end{tikzpicture}%
}

\newcommand{\gradientbox}[2]{%
    \BoxedText[draw=orange!70!black,right color=orange!10,left color=orange!50]     {start#1}{end#1}
    \tikzmark{start#1}#2
    \tikzmark{end#1}
}

\newcommand{\shadedbox}[2]{%
    \BoxedText[draw=cyan!70!black,fill=cyan!30,ultra thick]{start#1}{end#1}
    \tikzmark{start#1}#2
    \tikzmark{end#1}
}

\newcommand{\outlinebox}[2]{%
    \BoxedText{start#1}{end#1}
    \tikzmark{start#1}#2
    \tikzmark{end#1}
}

\newcommand{\sentence}[4]{$^{\circled{#2}}${\BoxedText[draw=#3!70!black,fill=#3!15,thick]{start#1}{end#1}
    \tikzmark{start#1}#4
    \tikzmark{end#1}}}

\usetikzlibrary{arrows}
\usetikzlibrary{shapes}
\newcommand{\indicator}[1]{%
  \tikz[baseline=(char.base)]\node[anchor=south west, draw,rectangle, rounded corners, inner sep=2pt, minimum size=6mm,
    text height=2mm](char){\textbf{#1}} ;}






\newcommand{\ptext}[1]{\parbox{.9\textwidth}{#1}}

\newcommand{\wff}
{Well Formed Formula }

\def\checkmark{\tikz\fill[scale=0.4](0,.35) -- (.25,0) -- (1,.7) -- (.25,.15) -- cycle;} 

%%%%%%%%%%%%%%%%%%%%%%%%%%%%%%%%%%%%%%%%%%%%%
\lhead{PHIL131.01 \\ Assignment-11}
\rhead{Atahan \\ Haznedar} 
\chead{05318867572 \\\textbf{2021107075}}
%%%%%%%%%%%%%%%%%%%%%%%%%%%%%%%%%%%%%%%%%%%%%


\begin{document}
    \section*{Chapter 8}
    \paragraph{Supplementary Problems}
    Rewrite the arguments below in standard form, and discuss the fallacy or fallacies (if any) committed in each. If there is no explicit argument but the passage seems to suggest one, write the suggested argument in standard form and evaluate it.\\[2pt]
    
    \begin{problem}{(28)}
        I should never watch the Yankees on TV. Whenever I do, they lose.
    \end{problem}
    
    \begin{solution}
        \begin{center}  
        \begin{tabular}{l l}
             &  Whenever I watch Yankees on TV, they lose \\\hline
             $\therefore$ &  I should never watch the Yankees on TV.
        \end{tabular}
        \end{center}
        
        There is a Post Hoc Fallacy and there is a Hasty Generalisation. There are insufficient number of samples because there is no correlation between watching and loss of a team. Also just something happens after another event does not make it a cause of it so there is also a Post Hoc Fallacy.
    \end{solution}
    
    \clearpage
    
    \begin{problem}{(29)}
        If the murder weapon was a gun, police should have found powder marks in the vicinity of the body.  The police found several powder marks on the carpet near where the victim fell. Therefore, the murder weapon was a gun.
    \end{problem}
    
    \begin{solution}
            \begin{center}
            
            \begin{tabular}{l l}
                 &  If the murder weapon was a gun, police should have found - \\ & powder marks in the vicinity of the body. \\
                 & The police found several powder marks on the carpet near - \\
                 & where the victim fell. \\ \hline
                 $\therefore$ & The murder weapon was a gun.
            \end{tabular}
            \end{center}
        
        This is a Formal Fallacy called "Affirming The Consequent" since the argument has a form of $$ P\rightarrow Q, Q \vdash P $$
    \end{solution}
    
\clearpage    
    
    \begin{problem}{(31)}
        Logicians claim that the study of argument is an indispensable part of everyone’s education. Since this helps to keep them employed, it’s not surprising that they endorse such views. Let’s not be influenced by thinking which does not represent the majority of students, parents, or teachers themselves.     
    \end{problem}
    
    \begin{solution}
            \begin{center}
                \begin{tabular}{l l}
                     &  Logicians claim that the study of argument is an indispensable-\\
                     & part of everyone’s education.\\
                     & Since this helps to keep them employed, it’s not surprising- \\
                     & that they endorse such views.  \\ \hline
                   $\therefore$  & Let’s not be influenced by thinking which does not represent -\\ 
                   & the majority of students, parents, or teachers themselves 
                \end{tabular}
            \end{center}
            
            This argument commits "Vested Interest Fallacy". Arguer thinks and believes that logicians does not represent the majority of society so their opinions shouldn't be listened. 
    \end{solution}
    

    
    \clearpage
    
 
    \begin{problem}{(32)}
        Timmy the Sheik thinks the Bears will repeat in the Super Bowl. So I’ve already asked my bookie  to put \$500 on the Bears next January. By the way, did you catch Timmy’s one-minute spot for the American Cancer Association? He’s right. The odds are against you if you smoke. I’m quitting cigarettes today.         
    \end{problem}
    
    \begin{solution}
            \begin{center}
                \begin{tabular}{l l}
                     & Timmy the Sheik thinks the Bears will repeat in the Super Bowl. \\ \hline
                     $\therefore$ & so I’ve  asked my bookie  to put \$500 on the Bears next January. \\
                     & \\
                     & \\
                     & Timmy was right about one-minute spot for the ACA \\
                     & He’s right. The odds are against you if you smoke. \\ \hline
                     $\therefore$ & I’m quitting cigarettes today.         
                \end{tabular}
            \end{center}
            
            
            In both of the atrguments there is a fallacy called Appeal to Authority. There is no reason in argument that Timmy must be listened to. 
    \end{solution}
    
\clearpage    
    
    \begin{problem}{(33)}
        An exhaustive survey of the class of 2000 revealed that 81 percent believe in God, 91 percent believe in the importance of family life, and 95 percent respect both of their parents. This survey also showed that over three-fourths (77 percent) of entering students placed ‘‘having a professional career’’ high on the list of their objectives. Clearly, the return to traditional American moral values is responsible for the new vocational emphasis among our college-age youth.        
    \end{problem}

    \begin{solution}
            \begin{center}
                \begin{tabular}{l l}
                     &  An exhaustive survey of the class of 2000 revealed that 81 -\\
                     & percent believe in God, 91 percent believe in the importance of -\\
                     & family life, and 95 percent respect both of their parents.  -\\
                     & This survey also showed that over three-fourths (77 percent) -\\
                     & of entering students placed ‘‘having a professional career’’-\\
                     &high on the list of their objectives.\\  \hline
                     $\therefore$& Clearly, the return to traditional American moral values is responsible - \\
                     & for the new vocational emphasis among our college-age youth.
                \end{tabular}
            \end{center}
            
            There is a fallacy of suppressed premises. There is insufficient premises to conclude something. Although 2000 samples might be a good number of samples there should be more categories to ask for to conclude a big generalisation.
    \end{solution}
    
\clearpage

    \begin{problem}{(36) }
        In the ruin of a family, its immemorial laws perish; and when the laws perish, the whole family is overcome by lawlessness. And when lawlessness prevails, . . . the women of the family are corrupted, . . . [and] a mixture of caste arises. . . . And this confusion brings the family itself to hell and those who have destroyed it; for their ancestors fall, deprived of their offerings of rice and water. By the sins of those who destroy a family and create a mixture of caste, the eternal laws of the caste and the family are destroyed. (The Bhagavad-Gita, I, 40–43)
    \end{problem}
    
    \begin{solution}
            \begin{center}
                \begin{tabular}{l l}
                     &  In the ruin of a family, its immemorial laws perish;\\
                     &  When the laws perish, the whole family is overcome by lawlessness. \\
                     & And when lawlessness prevails, . . . the women of the family are corrupted,\\
                     & . . . [and] a mixture of caste arises. . . . \\
                     & And this confusion brings the family itself to hell and those who have destroyed it; \\
                     & for their ancestors fall, deprived of their offerings of rice and water.\\ \hline
                     $\therefore$& By the sins of those who destroy a family and create a mixture of caste,\\
                     & the eternal laws of the caste and the family are destroyed.
                \end{tabular}
            \end{center}
            
            There is a "Slippery Slope Fallacy." There are lots of causal relationship in the argument which does not neceserrially affect each other. Hence overall a fallacy is made.
    \end{solution}
    
\clearpage

    \begin{problem}{(39)}
        While the nuclear power plant has been shut down since the last radiation leak, local residents are still getting their electricity from alternative sources. The good weather and pretty scenery in the area, plus ideal shopping and recreation, mean that citizens should not worry excessively about the on again, off again status of a single reactor.
    \end{problem}
    
    \begin{solution}
     \begin{center}
         \begin{tabular}{l l}
              &  While the nuclear power plant has been shut down since the -\\
              & last radiation leak, local residents are still getting their electricity- \\
              & from alternative sources.\\ \hline
              $\therefore$& The good weather and pretty scenery in the area, plus ideal shopping- \\
              & and recreation, mean that citizens should not worry excessively - \\
              & about the on again, off again status of a single reactor.
         \end{tabular}
     \end{center}
            
            
            There is a fallacy of "red herring" and "missing the point". The premise and the conclusion are irrevalant.
            
    \end{solution}
    
\end{document}