\documentclass[12pt]{article}
\usepackage[margin=5cm]{geometry} 
\usepackage{amsmath,amsthm,amssymb,amsfonts, fancyhdr, color, comment, graphicx, environ}
\usepackage{xcolor}
\usepackage{mdframed}
\usepackage[shortlabels]{enumitem}
\usepackage{indentfirst}
\usepackage{hyperref}
\usepackage{tensor}

\usepackage{logicproof}

\usepackage{graphicx}
\graphicspath{ {./images/} }

\hypersetup{
    colorlinks=true,
    linkcolor=blue,
    filecolor=magenta,      
    urlcolor=blue,
}
\usepackage{tikz-cd}
\tikzset{ampersand replacement = \&}
\pagestyle{fancy}

\usepackage{tikz}
\newcommand*\circled[1]{\tikz[baseline=(char.base)]{\node[shape=circle,draw,inner sep=1pt] (char) {#1};}}
            
    

\newenvironment{problem}[2][Problem]
    { \begin{mdframed}[backgroundcolor=gray!20] \textbf{#1 #2} \\}
    {  \end{mdframed}}

\newenvironment{solution}
    {{\textbf{Solution}}}

\def\boxit#1#2{%
    \smash{\color{red}\fboxrule=1pt\relax\fboxsep=2pt\relax%
    \llap{\rlap{\fbox{\phantom{\rule{#1}{#2}}}}~}}\ignorespaces
}

\newcommand\LL[1]{\multicolumn{1}{|c}{#1}}
\newcommand\RR[1]{\multicolumn{1}{c|}{#1}}
\newcommand\LR[1]{\multicolumn{1}{|c|}{#1}}
% \cline{2-7}


%%%%%%%%%%%%%%%%%%%%%%%%%%%%%%%%%%%%%%%%%%%%%%%
\usepackage{tikzpagenodes}
\usetikzlibrary{calc}

\makeatletter
\tikzset{%
  remember picture with id/.style={%
    remember picture,
    overlay,
    save picture id=#1,
  },
  save picture id/.code={%
    \edef\pgf@temp{#1}%
    \immediate\write\pgfutil@auxout{%
      \noexpand\savepointas{\pgf@temp}{\pgfpictureid}}%
  },
  if picture id/.code args={#1#2#3}{%
    \@ifundefined{save@pt@#1}{%
      \pgfkeysalso{#3}%
    }{
      \pgfkeysalso{#2}%
    }
  }
}

\def\savepointas#1#2{%
  \expandafter\gdef\csname save@pt@#1\endcsname{#2}%
}

\def\tmk@labeldef#1,#2\@nil{%
  \def\tmk@label{#1}%
  \def\tmk@def{#2}%
}

\tikzdeclarecoordinatesystem{pic}{%
  \pgfutil@in@,{#1}%
  \ifpgfutil@in@%
    \tmk@labeldef#1\@nil
  \else
    \tmk@labeldef#1,(0pt,0pt)\@nil
  \fi
  \@ifundefined{save@pt@\tmk@label}{%
    \tikz@scan@one@point\pgfutil@firstofone\tmk@def
  }{%
  \pgfsys@getposition{\csname save@pt@\tmk@label\endcsname}\save@orig@pic%
  \pgfsys@getposition{\pgfpictureid}\save@this@pic%
  \pgf@process{\pgfpointorigin\save@this@pic}%
  \pgf@xa=\pgf@x
  \pgf@ya=\pgf@y
  \pgf@process{\pgfpointorigin\save@orig@pic}%
  \advance\pgf@x by -\pgf@xa
  \advance\pgf@y by -\pgf@ya
  }%
}
\newcommand\tikzmark[2][]{%
\tikz[remember picture with \usepackage{logicproof}
id=#2] #1;}
\makeatother

\newcommand\BoxedText[3][]{%
\begin{tikzpicture}[remember picture,overlay]
\draw[#1] 
  let \p1=(pic cs:#2), \p2=(pic cs:#3) in
  ([yshift=-0.8ex]\p1) --
  ([yshift=2ex]\p1) -- 
  ([xshift=3pt,yshift=2ex]\p1-|current page text area.east) -- 
  ([xshift=3pt,yshift=2ex]\p2-|current page text area.east) --
  ([yshift=2ex]\p2) --
  ([yshift=-0.8ex]\p2) --
  ([xshift=-3pt,yshift=-0.8ex]\p2-|current page text area.west) --
  ([xshift=-3pt,yshift=-0.8ex]\p1-|current page text area.west) --
  cycle
;
\end{tikzpicture}%
}

\newcommand{\gradientbox}[2]{%
    \BoxedText[draw=orange!70!black,right color=orange!10,left color=orange!50]     {start#1}{end#1}
    \tikzmark{start#1}#2
    \tikzmark{end#1}
}

\newcommand{\shadedbox}[2]{%
    \BoxedText[draw=cyan!70!black,fill=cyan!30,ultra thick]{start#1}{end#1}
    \tikzmark{start#1}#2
    \tikzmark{end#1}
}

\newcommand{\outlinebox}[2]{%
    \BoxedText{start#1}{end#1}
    \tikzmark{start#1}#2
    \tikzmark{end#1}
}

\newcommand{\sentence}[4]{$^{\circled{#2}}${\BoxedText[draw=#3!70!black,fill=#3!15,thick]{start#1}{end#1}
    \tikzmark{start#1}#4
    \tikzmark{end#1}}}

\usetikzlibrary{arrows}
\usetikzlibrary{shapes}
\newcommand{\indicator}[1]{%
  \tikz[baseline=(char.base)]\node[anchor=south west, draw,rectangle, rounded corners, inner sep=2pt, minimum size=6mm,
    text height=2mm](char){\textbf{#1}} ;}






\newcommand{\ptext}[1]{\parbox{.9\textwidth}{#1}}

\newcommand{\wff}
{Well Formed Formula }

\def\checkmark{\tikz\fill[scale=0.4](0,.35) -- (.25,0) -- (1,.7) -- (.25,.15) -- cycle;} 

%%%%%%%%%%%%%%%%%%%%%%%%%%%%%%%%%%%%%%%%%%%%%
\lhead{PHIL131.01 \\ Assignment-9}
\rhead{Atahan \\ Haznedar} 
\chead{05318867572 \\\textbf{2021107075}}
%%%%%%%%%%%%%%%%%%%%%%%%%%%%%%%%%%%%%%%%%%%%%


\begin{document}
    \section*{Chapter 6}
    \paragraph{II}
    Determine which of the following formulas are wffs and which are not. Explain your answer. \\[2pt]
    
\begin{problem}{(4)}
    $$\sim F_{xy}$$
\end{problem}    
    
\begin{solution}
    Since x and y lacks a quantifier it not a \wff.
\end{solution}

\begin{problem}{(8)}
    $$\forall x F_x \rightarrow L_{ax}$$
\end{problem}

\begin{solution}
    There should be outer brackets outside the implication. So it is not a \wff.
\end{solution}
    
\begin{problem}{(10)}
    $$ \forall x \forall y (F_x \rightarrow \sim y = x)$$
\end{problem}    

\begin{solution}
    '$F_a$' is a wff, by rule 1; and since '$b=a$' is a a wff by the rule for identity, '$\sim b=a$' is a wff, by rule 2. Hence '$(F_a \rightarrow \sim b = a)$' is a wff, by rule 3, when two applications of rule 4 it follows that '$\forall x \forall y (F_x \rightarrow \sim y = x)$' is a wff.
\end{solution}
    
    \clearpage
    \paragraph{III}    
    Evaluate the following wffs in the model given below: \\[0.5cm]


    \begin{tabular}{l l}
         Universe: & the states of the U.S. \\
         c: & California \\
         n: & Nevada \\
         u: & Utah \\
         E: & the class of states on the East Coast \\
         A: & the relation that holds between two adjacent states.\\
         C: & : the relation that holds between three states x, y, and z\\
         & just in case the capital of x is closer to the capital of y \\
          & than to the capital of z.
    \end{tabular}
    
\begin{problem}{(7)}
    $$\exists x (E_x \& A_{xx} )$$
\end{problem}

\begin{solution}
    Since something cannot be adjacent with itself therefore it is False.
\end{solution}

\begin{problem}{(12)}
    $$\forall x \exists y \exists z C_{xyz}$$
\end{problem}

\begin{solution}
    If y is the same as z so the capital of them should be same so it can't be closer, therefore it is false.
\end{solution}

\begin{problem}{(18)}
    $$ \forall x \forall y (A_{xy} \iff A_{yz})$$
\end{problem}

\begin{solution}
    Since adjacent property is commutative it is True.
\end{solution}





    \clearpage
    \paragraph{IV}
    Determine which of the following argument forms is valid by examining whether there exists a model in which the premises are true and the conclusion false.
    
    \begin{problem}{(6)}
        $$ \forall x \sim F_x \vdash \sim \exists x F_x $$
    \end{problem}
    
    \begin{solution}
        It is valid since we know the negation of a quantifier changes the quantifier and negates the inside. Therefore it is valid.
    \end{solution}
    
    \begin{problem}{(8)}
        $$\sim \exists x F_x \vdash \forall x \sim F_x$$ 
    \end{problem}
    
    \begin{solution}
        It is valid since we know the negation of a quantifier changes the quantifier and negates the inside. Therefore it is valid.
    \end{solution}
    
    \begin{problem}{(9)}
        $$ \sim \forall x F_x \vdash \exists x \sim F_x $$    
    \end{problem}
    
    \begin{solution}
        It is valid since we know the negation of a quantifier changes the quantifier and negates the inside. Therefore it is valid.
    \end{solution}
    
    \clearpage
    \paragraph{VI}
    Formalize each of the following arguments, using the interpretation given below.
    
    \begin{tabular}{c|c}
         Symbols & Interpretation  \\ \hline
         Names &  \\
         p & propositional logic \\
         r & predicate logic \\
         i & predicate logic with identity \\
         One-Place Predicates & \\
         R & is a set of rules \\
         S & is a formal system \\
         Two-Place Predicates & \\
         F & is a formula of \\
         P & is a part of \\
         W & is a wff of \\
    \end{tabular}
    
\begin{problem}{(1)}
    Propositional logic is a part of predicate logic. Therefore, predicate logic is not a part of propositional logic.
\end{problem}

\begin{solution}
    $$P_{pr} \vdash \sim P_{rp} $$
\end{solution}

\begin{problem}{(5)}
    Every formal system is a set of rules. Therefore, every set of rules is a formal system
\end{problem}

\begin{solution}
    $$ \forall x (S_x \rightarrow R_x) \vdash \forall x (R_x \rightarrow S_x)$$
\end{solution}

\clearpage

\begin{problem}{(14)}
    If one formal system is part of a second formal system, then every wff of the first is a wff of the second. Predicate logic is a part of predicate logic with identity, and both are formal systems. Therefore, every wff of predicate logic is also a wff of predicate logic with identity.
\end{problem}

\begin{solution}
    $$\exists x, y (((S_x \& S_y) \& P_{xy})\rightarrow (\forall w (W_{wx} \rightarrow  W_{wy}))), (P_{ri} \& (S_r \& S_i)) \vdash \forall w (W_{wr} \rightarrow W_{wi})$$
\end{solution}


\end{document}