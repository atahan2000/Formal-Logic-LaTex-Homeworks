\documentclass[12pt]{article}
\usepackage[margin=5cm]{geometry} 
\usepackage{amsmath,amsthm,amssymb,amsfonts, fancyhdr, color, comment, graphicx, environ}
\usepackage{xcolor}
\usepackage{mdframed}
\usepackage[shortlabels]{enumitem}
\usepackage{indentfirst}
\usepackage{hyperref}
\usepackage{tensor}

\usepackage{logicproof}



\hypersetup{
    colorlinks=true,
    linkcolor=blue,
    filecolor=magenta,      
    urlcolor=blue,
}
\usepackage{tikz-cd}
\tikzset{ampersand replacement = \&}
\pagestyle{fancy}

\usepackage{tikz}
\newcommand*\circled[1]{\tikz[baseline=(char.base)]{\node[shape=circle,draw,inner sep=1pt] (char) {#1};}}
            
    

\newenvironment{problem}[2][Problem]
    { \begin{mdframed}[backgroundcolor=gray!20] \textbf{#1 #2} \\}
    {  \end{mdframed}}

\newenvironment{solution}
    {{\textbf{Solution}}}

\def\boxit#1#2{%
    \smash{\color{red}\fboxrule=1pt\relax\fboxsep=2pt\relax%
    \llap{\rlap{\fbox{\phantom{\rule{#1}{#2}}}}~}}\ignorespaces
}

\newcommand\LL[1]{\multicolumn{1}{|c}{#1}}
\newcommand\RR[1]{\multicolumn{1}{c|}{#1}}
\newcommand\LR[1]{\multicolumn{1}{|c|}{#1}}
% \cline{2-7}


%%%%%%%%%%%%%%%%%%%%%%%%%%%%%%%%%%%%%%%%%%%%%%%
\usepackage{tikzpagenodes}
\usetikzlibrary{calc}

\makeatletter
\tikzset{%
  remember picture with id/.style={%
    remember picture,
    overlay,
    save picture id=#1,
  },
  save picture id/.code={%
    \edef\pgf@temp{#1}%
    \immediate\write\pgfutil@auxout{%
      \noexpand\savepointas{\pgf@temp}{\pgfpictureid}}%
  },
  if picture id/.code args={#1#2#3}{%
    \@ifundefined{save@pt@#1}{%
      \pgfkeysalso{#3}%
    }{
      \pgfkeysalso{#2}%
    }
  }
}

\def\savepointas#1#2{%
  \expandafter\gdef\csname save@pt@#1\endcsname{#2}%
}

\def\tmk@labeldef#1,#2\@nil{%
  \def\tmk@label{#1}%
  \def\tmk@def{#2}%
}

\tikzdeclarecoordinatesystem{pic}{%
  \pgfutil@in@,{#1}%
  \ifpgfutil@in@%
    \tmk@labeldef#1\@nil
  \else
    \tmk@labeldef#1,(0pt,0pt)\@nil
  \fi
  \@ifundefined{save@pt@\tmk@label}{%
    \tikz@scan@one@point\pgfutil@firstofone\tmk@def
  }{%
  \pgfsys@getposition{\csname save@pt@\tmk@label\endcsname}\save@orig@pic%
  \pgfsys@getposition{\pgfpictureid}\save@this@pic%
  \pgf@process{\pgfpointorigin\save@this@pic}%
  \pgf@xa=\pgf@x
  \pgf@ya=\pgf@y
  \pgf@process{\pgfpointorigin\save@orig@pic}%
  \advance\pgf@x by -\pgf@xa
  \advance\pgf@y by -\pgf@ya
  }%
}
\newcommand\tikzmark[2][]{%
\tikz[remember picture with \usepackage{logicproof}
id=#2] #1;}
\makeatother

\newcommand\BoxedText[3][]{%
\begin{tikzpicture}[remember picture,overlay]
\draw[#1] 
  let \p1=(pic cs:#2), \p2=(pic cs:#3) in
  ([yshift=-0.8ex]\p1) --
  ([yshift=2ex]\p1) -- 
  ([xshift=3pt,yshift=2ex]\p1-|current page text area.east) -- 
  ([xshift=3pt,yshift=2ex]\p2-|current page text area.east) --
  ([yshift=2ex]\p2) --
  ([yshift=-0.8ex]\p2) --
  ([xshift=-3pt,yshift=-0.8ex]\p2-|current page text area.west) --
  ([xshift=-3pt,yshift=-0.8ex]\p1-|current page text area.west) --
  cycle
;
\end{tikzpicture}%
}

\newcommand{\gradientbox}[2]{%
    \BoxedText[draw=orange!70!black,right color=orange!10,left color=orange!50]     {start#1}{end#1}
    \tikzmark{start#1}#2
    \tikzmark{end#1}
}

\newcommand{\shadedbox}[2]{%
    \BoxedText[draw=cyan!70!black,fill=cyan!30,ultra thick]{start#1}{end#1}
    \tikzmark{start#1}#2
    \tikzmark{end#1}
}

\newcommand{\outlinebox}[2]{%
    \BoxedText{start#1}{end#1}
    \tikzmark{start#1}#2
    \tikzmark{end#1}
}

\newcommand{\sentence}[4]{$^{\circled{#2}}${\BoxedText[draw=#3!70!black,fill=#3!15,thick]{start#1}{end#1}
    \tikzmark{start#1}#4
    \tikzmark{end#1}}}

\usetikzlibrary{arrows}
\usetikzlibrary{shapes}
\newcommand{\indicator}[1]{%
  \tikz[baseline=(char.base)]\node[anchor=south west, draw,rectangle, rounded corners, inner sep=2pt, minimum size=6mm,
    text height=2mm](char){\textbf{#1}} ;}






\newcommand{\ptext}[1]{\parbox{.9\textwidth}{#1}}

\newcommand{\wff}
{Well Formed Formula }

\def\checkmark{\tikz\fill[scale=0.4](0,.35) -- (.25,0) -- (1,.7) -- (.25,.15) -- cycle;} 

%%%%%%%%%%%%%%%%%%%%%%%%%%%%%%%%%%%%%%%%%%%%%
\lhead{PHIL131.01 \\ Assignment-6}
\rhead{Atahan \\ Haznedar} 
\chead{\textbf{2021107075}}
%%%%%%%%%%%%%%%%%%%%%%%%%%%%%%%%%%%%%%%%%%%%%


\begin{document}
    \section*{Chapter 5}
    \paragraph{III}
    Using basic or derived rules, prove the validity of the following argument forms.
    
    \begin{problem}{1}
        $P \iff Q,Q\iff R \vdash P \iff R$
    \end{problem}
    
    \begin{solution} 
        \begin{logicproof}{1}
            P \iff Q & A \\
            (P \Rightarrow Q) \& (Q \Rightarrow P) & 1\iff E\\
            P \Rightarrow Q & 2\&E \\
            Q \Rightarrow P & 2\&E \\
            Q \iff R & A \\
            (Q \Rightarrow R )\&(R \Rightarrow Q) & 5\iff E \\
            Q \Rightarrow R & 6\&E \\
            R \Rightarrow Q & 6\&E \\
                \begin{subproof}
                    P & H \\
                    Q & 9,3\Rightarrow E \\
                    R & 10,7\Rightarrow E
                \end{subproof}
            P \Rightarrow R & 9-11\Rightarrow I \\
                \begin{subproof}
                    R & H \\
                    Q & 13,8\Rightarrow E \\
                    P & 14,4\Rightarrow E 
                \end{subproof}
            R \Rightarrow P & 13-15 \Rightarrow I \\
            P \iff R & 12,16\iff I
        \end{logicproof}
    \end{solution}

\clearpage

    \begin{problem}{2}
        $P \Rightarrow Q \vdash (P\& R) \Rightarrow (Q \& R)$
    \end{problem}
    
    \begin{solution}
        \begin{logicproof}{2}
            P \Rightarrow Q & A \\
                \begin{subproof}
                    P \& R & H \\
                    P & 2 \&E \\
                    R & 2\& E \\
                    Q & 1,3 \Rightarrow E \\
                    Q \& R & 4,5 \& I
                \end{subproof}
            (P\&R)\Rightarrow(Q\&R) & 2-6\Rightarrow I
        \end{logicproof}
    \end{solution}


\paragraph{IV}
Using basic or derived rules, prove the following theorems.

    \begin{problem}{4}
        $(P\rightarrow Q) \rightarrow (\sim Q \rightarrow \sim P)$
    \end{problem}
    
    \begin{solution}
        \begin{logicproof}{3}
            \begin{subproof}
                $P \rightarrow Q$ & H \\
                    \begin{subproof}
                        $\sim Q$ & H \\
                            \begin{subproof}
                                P & H \\
                                Q & $1,3 \rightarrow E$\\
                                $\sim Q$ \& Q & 2,4 \& I  
                            \end{subproof}
                        $\sim$ P & $3-5 \sim I$ 
                    \end{subproof}
                    $\sim Q \rightarrow \sim P$ & $2-6\rightarrow I$ 
            \end{subproof}
        $(P \rightarrow Q)\rightarrow(\sim Q \rightarrow \sim P)$ & $1-7 \rightarrow I$    
        \end{logicproof}
    \end{solution}
\clearpage

    \begin{problem}{8}
        $\vdash P \lor (P \rightarrow Q)$
    \end{problem}
    
    \begin{solution}
        \begin{logicproof}{4}
            \begin{subproof}
            $\sim (P \lor (P \rightarrow Q))$ & H \\
            $\sim P \& \sim (P \rightarrow Q)$ & 1DM \\
            $\sim (P \rightarrow Q)$ & $2 \& E$ \\
            $\sim (\sim P \lor Q)$ & 3MI \\
            $\sim \sim P \& \sim Q$ & 4DM \\
            $P \& \sim Q$ & 5DN \\
            $P$  & 6\&E \\
            $\sim P$ &  2 \& E \\
            $P \& \sim P$ & $7,8 \& I$ 
            \end{subproof}
            $\sim \sim (P \lor (P \rightarrow Q))$ & $1-9\sim I$ \\
            $(P \lor (P \rightarrow Q))$ & 10DN
        \end{logicproof}
    \end{solution}
    
\clearpage
\paragraph{V}
Using basis or derived rules, prove the following equivalences.
    
    \begin{problem}{1} 
        $\vdash (P \& Q) \iff \sim (\sim P \lor \sim Q)$
    \end{problem}
    
    \begin{solution}
        \begin{logicproof}{5}
            \begin{subproof}
            $P \& Q$ & H \\
            P & 1\&E \\
            Q & 1\&E \\
                \begin{subproof}
                \sim P \lor \sim Q & H \\ 
                \sim(P \& Q) & 4DM \\
                (P \& Q) \& \sim (P \& Q) & 5,1\&I 
                \end{subproof}
            $\sim (\sim P \lor \sim Q)$ & $4-6 \sim I$
            \end{subproof}
        $(P \& Q) \rightarrow \sim (\sim P \lor \sim Q)$ & $1-7 \rightarrow I$\\
            \begin{subproof}
                $\sim (\sim P \lor \sim Q)$ & H \\
                $\sim \sim P \& \sim \sim Q$ & 9DM \\
                $\sim \sim P$ & 10 \& E \\
                P & 11DN \\
                $\sim \sim Q$ & 10 \& E \\
                Q & 13DN \\
                $P\&Q$ & 12,14\&I 
            \end{subproof}
        $(\sim (\sim P \lor \sim Q)) \rightarrow (P \& Q)$ & 9-15 \rightarrow I \\
        $(P \& Q) \iff (\sim (\sim P \lor \sim Q))$ & 8,16 \iff I 
        \end{logicproof}    
    \end{solution}
 \clearpage
 
    \begin{problem}{4}
        $\vdash (P \lor Q) \iff \sim P \rightarrow Q$
    \end{problem}
    
    \begin{solution}
        \begin{logicproof}{6}
            \begin{subproof}
                $P \lor Q$ & H \\
                \begin{subproof}
                    $\sim P$ & H \\
                    $(P \lor Q) \& \sim P$ & $1,3 \& I$ \\
                    $(P \& \sim P) \lor (\sim P \& Q)$ & 3DIST \\
                    $\sim P \& Q$ & 4CONT \\
                    $Q$ & $5 \& E$ 
                \end{subproof}
            $\sim P \rightarrow Q$ & $2-6 \rightarrow I$ 
            \end{subproof}
        $(P \lor Q) \rightarrow (\sim P \rightarrow Q)$ & $1-7 \rightarrow I$ \\
            \begin{subproof}
            $\sim P \rightarrow Q$ & H \\
            $\sim \sim P \lor Q$ & 9MI \\
            $P \lor Q$ & 10DN 
            \end{subproof}
        $(\sim P \rightarrow Q) \rightarrow (P \lor Q)$ & $9-11\rightarrow I$ \\
        $(P \lor Q) \iff (\sim P \rightarrow Q)$ & $8,12 \iff I$
        \end{logicproof}
\end{document}